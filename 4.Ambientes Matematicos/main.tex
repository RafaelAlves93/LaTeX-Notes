\documentclass[12pt,a4paper]{article}
%\documentclass[a4paper,12pt]{report}
\usepackage[utf8]{inputenc} % Pacote para acentuação
\usepackage[portuguese]{babel} % Para quem vai escrever em PT-PT
\usepackage[a4paper,lmargin=2cm,tmargin=1cm,rmargin=2cm,bmargin=2cm]{geometry}
%\usepackage[lmargin=2cm,tmargin=2cm,rmargin=2cm,bmargin=2cm]{geometry}
\usepackage{indentfirst} % Identacao primeiro paragrafo
\usepackage{amsmath,amsthm,amsfonts,amssymb,dsfont,mathtools} % Pacotes matematicos



\title{Ambientes Matemáticos}
\author{Rafael Alves}
\date{\today}

\begin{document}
\maketitle

\section{Criar uma Matriz}
\subsection{Com Parêntesis}

$$  % TUDO O QUE FOR MATEMATICO TEM DE TER ANTES E DEPOIS $
\left(\begin{array}{ccc}
    a   &   b   &   f\\
    c   &   d   &   g\\
    h   &   i   &   j
\end{array}\right)
$$

\subsection{Com Chavetas Retas}
$$  % TUDO O QUE FOR MATEMATICO TEM DE TER ANTES E DEPOIS $
\left[\begin{array}{ccc}
    a   &   b   &   f\\
    c   &   d   &   g\\
    h   &   i   &   j
\end{array}\right]
$$

\subsection{Determinante}
$$
\left|\begin{array}{ccc}
    a   &   b   &   f\\
    c   &   d   &   g\\
    h   &   i   &   j
\end{array}\right|
$$

\subsection{Sistema}
\vspace{1cm}
$$
\left\{\begin{array}{ccc}
    a   &   b   &   f\\
    c   &   d   &   g\\
    h   &   i   &   j
\end{array}\right\}
$$
\vspace{1cm}

\section{Equações Matemáticas}
\begin{center}
$$
\left(\begin{array}{ccc}
    \frac{1}{2}     &       b       &   x^2\\
    c               &   \sqrt{4}    &   g\\
    \sqrt[3]{8}     &       i       &   \displaystyle\lim_{x\rightarrow 0}f(x)
\end{array}\right)
$$
\end{center}


\begin{equation}
    y=x^2+1
\end{equation}

\begin{equation}
    \displaystyle\int_0^1 f(x)dx
\end{equation}


\section{Elementos Matemáticos}

\begin{itemize} % OU ENUMERATE
    \item [] 1) Sinal da Soma: + ou $+$
    \item [] 2) Sinal da Diferença: - ou $-$
    \item [] 3) Sinal da Multiplicação: $x\cdot y$ ou $x\times y$
    \item [] 4) Sinal da Divisão: $\frac{1}{2}$ ou $\dfrac{1}{2}$ ou $1\div 2$
\end{itemize}


\vspace{2cm}
A fórmula fundamental da trigonometria, onde surgem os valores de $\sin \alpha$ e de $\cos \alpha$, apresenta o seguinte resultado:

\begin{equation}
    \sin^2 \alpha + \cos^2 \alpha =1
    \label{fft}
\end{equation}

A equação \ref{fft} pode ser alterada dividindo todos os termos por:
\[\cos^2 \alpha\]

resultando em:

$$
\tan^2 \alpha +1 = \frac{1}{\cos^2\alpha}
$$

 
\end{document}
